\documentclass[10pt]{article}

\usepackage[numbers]{natbib}
\usepackage{microtype}
\usepackage{graphicx}
\usepackage{fullpage}
\usepackage{color}
\usepackage{url}
\usepackage[pdftex]{hyperref}
\usepackage{wasysym}

\newcommand{\textsec}[1]{\textsection\ref{#1}}

\begin{document}
\thispagestyle{empty}

\hrule
\vspace{1mm}
\noindent\begin{tabular}{ll}
\bf Title   & \textbf{Project Title} \\
\bf Course  & CS571: Natural Language Processing \\
\bf Authors & Name 1, 3rd-year Undergraduate in Computer Science, \texttt{name1@emory.edu} \\ 
            & Name 2, 1st-year Ph.D. student in Computer Science, \texttt{name2@emory.edu} \\ 
\bf Advisor & Jinho D.\ Choi, Assistant Professor of Computer Science, \texttt{jinho.choi@emory.edu}\\
\end{tabular}
\vspace{1mm}
\hrule

% ============================== ABSTRACT ==============================

\section{Abstract}

\textit{To be filled.}

% ============================== INTELLECTUAL MERIT ==============================

\section{Intellectual Merit}

\textit{To be filled.}

% ============================== BROADER IMPACT ==============================

\section{Broader Impact}

\textit{To be filled.}

% ============================== OBJECTIVES ==============================

\section{Introduction}

\textit{Objectives, motivation, problem statements, etc.}

% ============================== BACKGROUND ==============================

\section{Background}

\subsection{Related Work}

\textit{To be filled.}
This is how you cite~\cite{choi:16a}.

\subsection{Preliminary Work}

\textit{Related work done by the authors.}


% ============================== PROPOSED RESEARCH ==============================

\section{Proposed Research}

\subsection{Approach}

\textit{To be filled.}

\subsection{Experiments}

\textit{Data to be used, evaluation methods, comparison against previous work (meaningful baseline must be specified), expected output, etc.}

\section{Timeline}

\textit{If group project, who will do what task in which period.}


\bibliographystyle{plainnat}
\bibliography{proposal}

\end{document}

